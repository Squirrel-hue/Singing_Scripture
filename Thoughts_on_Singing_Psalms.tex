\documentclass{article}

\title{Thoughts on Psalms in Christian Worship}

\usepackage{hyperref}

\begin{document}

\maketitle


\section{Motivation and Value of Singing Psalms}

How do we know that we are properly including the whole counsel of God as revealed in the Scripture? Even so, in what manner can be be sure that we are properly weighting the different themes in emphasis and adequately reflecting the different relationships between them.

There may be many ways of answering these questions, but one way would be to make use of the songs that God gave directly to his people.  The Psalms reflec 
How do we know that we are properly including, weighting, and capturing the most im

There is an old tradition, spanning centuries and languages --- Hebrew, Greek, Latin, French, English, Spanish --- of singing the Psalms in worship

\section{Scriptural Influence of Hymns and Songs}

Many of our songs are based on excerpts or verses and then these are expanded upon.

``Great is Thy Faithfulness" on Lamentations 3:23

``Holy, Holy, Holy" based on Isaiah 6 and Revelation 4:8 with verses drawn from the surrounding context. 

\section{Reasons to Sing Psalms}

The following reasons underlie this conviction:

\begin{itemize}
	\item Colossians 3:16 ``Let the word of Christ dwell in you richly, teaching and admonishing one another in all wisdom, singing \emph{psalms} and hymns and spiritual songs, with thankfulness to God." Note the all important context of the ``word of Christ dwell[ing] in your richly." It is the word of Christ that is to dwell in us, and after mentioning teaching, Paul then mentions psalms.  It may not include singing entire Psalms, but in no way would singing psalms violate this teaching of the apostle.  Never singing Psalms would seem to fall short by omission.
	\item That some verses or phrases from the Psalms are sung in many churches is undeniable, and this is a good thing.  But it may still fall short of the fulness of the meaning of the apostle, and we possess the means to approach or attain his exhortation in this matter.
	\item Nothing we think or write will equal (much less surpass) the gift of the Holy Spirit to the Church.
	\item In our worship, the closer we remain to Scripture the better, including even using the actual words or translated equivalents.
	\item It will aid in our memorization of Scripture.
	\item The Psalms are the most quoted book in the New Testament, and the allusions to it may go even further (allusions for example, Ps. 134 and Rev. 7:15; Ps 91? with Rev 7:15; Ps. 23 with Rev. 7:17.
	\item It will increase our witness and our confidence in our witness to the sufficiency of Scripture and testify to ourselves, visitors, younger believers, and unbelievers.  
	\item We will learn from God; we may not be able to explain why certain truths or matters are in the Bible, but we learn by encountering or experiencing the elements that are present in our life.  The Psalms have their origin in God, not in man, and through the life experiences of the various Psalmists, who spanned from at least time time of Moses until at least the time of Solomon during the Israel monarchy.)
\end{itemize}


There are undoubtedly other reasons given in the introductions to and throughout the following works:

\section{In English, Singing the Psalms is Very Accessible}

The Scottish Psalter of 1650 (available from the \href{https://archive.org/details/scotishpsalter/1650%20PSALTER/}{Internet Archive} and \href{https://thewestminsterstandard.org/1650-scottish-metrical-psalter/}{a page of the Westminster Standards}.  Due to the words of the Psalms being arranged in the Common Meter (8.6.8.6), this permits this Psalter to be sung to dozens of well-known tunes such as ``Amazing Grace" (technically, ``NEW BRITAIN") or ``O Little Town of Bethlehem."

\subsection{List of Tunes with Common Meter (8.6.8.6)}

It seems that the tradition is to capitalize tune title.

\begin{itemize}
	\item ANTIOCH (Joy to the World!)
	\item AURELIA
	\item AUSTRIA
	\item AVON (Alas, and Did My Savior Bleed)
	\item AZMON (O For a Thousand Tongues to Sing)
	\item CHRISTIAN LOVE
	\item CORONATION (All Hail the Power of Jesus' Name)
	\item DETROIT (Composer: Bradshaw, 1820)
	\item DOMINUS REGIT ME
	\item ELLACOMBE
	\item FOUNTAIN
	\item HAMBURG (When I Survey the Wondrous Cross)
	\item HYFRYDOL
	\item ITALIAN HYMN
	\item LAND OF REST
	\item LORD OF THE DANCE
	\item MACCABAEUS
	\item MARTYDROM
	\item MCKEE
	\item OLD HUNDREDTH
	\item REPTON
	\item SHANTI
	\item ST AGNES
	\item St. ANNE
	\item ST COLUMBA
	\item ST. PETER
	\item ST THEODULPH
	\item WINCHESTER OLD
	\item WOODLANDS
	\item and thousands more at \href{https://hymnary.org/search?page=3&qu=meter%3A8.6.8.6%20in%3Atunes&sort=totalInstances}{hymnary.org}
\end{itemize}

\section{List of Available Psalters}


\begin{itemize}

	\item Scottish Psalter
		\begin{itemize}
			\item 1564 Version
			\item \href{https://thewestminsterstandard.org/1650-scottish-metrical-psalter/}{1650 Scottish Metrical Psalter}
		\end{itemize}
	\item The Trinity Psalter
	\item The Book of Psalms for Singing
	\item The Book of Psalms for Worship
	\item The ARP Psalter With Bible Songs
	\item These last four references are included at the following link:
		\href{www.psalter.org}{Psalter.org} 
\end{itemize}



I am under the impression that the church in Geneva pastored by John Calvin heavily used the Psalms; what balance they had with other music (if any were present) is not clear to me.

I have not read (or cannot remember distinctly) any of Calvin's thoughts on this topic.

In no way is this intended to take away from the use of other hymns and spiritual songs, or even to suggest that a fixed number of Psalms must be sung every Lord's Day.  Still, I doubt any church would go too far astray by singing a Psalm or at least several consecutive verses of a Psalm each Lord's Day.

\end{document}

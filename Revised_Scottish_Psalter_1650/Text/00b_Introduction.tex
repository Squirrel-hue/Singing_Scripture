\subsection*{Comments}

The main purpose is to update the language forma some archaic language.

Keeping the meter is important; rhyming is optional.

The distinction between the second person singular and second person plural is maintained as it is a characteristic of the Hebrew. The table below summarizes the forms.

The forms suggest the relationship between the individual worshipper and the corporate aspects of worship.

All the Psalms are set in Common Meter (8.6.8.6); some Psalms have an additional version in another meter.

The texts of the Psalms are from https://thewestminsterstandard.org/1650-scottish-metrical-psalter/.

Examples of tunes in common meter include:

\begin{itemize}
\sc\item Amazing Grace
  \item Aberdeen
	\item Antioch (\textup{Joy to the World!}) 
	\item Arlington
	\item Avon (Alas, and Did My Savior Bleed)
	\item Azmon (\textup{O For a Thousand Tongues to Sing})
	\item Bradford
	\item Bristol
	\item Caddo
	\item Clinton
	\item Christian Love
	\item Coronation \textup{(All Hail the Power of Jesus' Name)}
	\item Detroit (Composer: Bradshaw, 1820)
	\item Dominus Regit Me
	\item Dundee
	\item Ellacombe
	\item Fountain
	\item Gloria Patri
	\item Hamburg (\textup{When I Survey the Wondrous Cross})
  \item I Sing the Almight Power of God
	\item Land of Rest
	\item Lord of the Dance
	\item Maccabaeus
	\item Manoah
	\item Martyrdom (\textup{Alas! and Did My Savior Bleed})
	\item McKee
	\item Naomi
	\item Pisgah
	\item Prayer
	\item Repton
	\item Richmond
	\item Shanti
	\item St. Agnes
	\item St. Anne
	\item St. Columba
	\item St. Etheldreda
	\item St. Peter
	\item St. Stephen
	\item St. Theodulph
	\item Walsall
	\item Winchester Old
	\item Woodlands
	\item \textup{and thousands more at \href{https://hymnary.org/search?page=3&qu=meter\%3A8.6.8.6\%20in\%3Atunes&sort=totalInstances}{hymnary.org}}
\end{itemize}


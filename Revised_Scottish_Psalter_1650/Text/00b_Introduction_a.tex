\subsection*{Comments}

The main purpose is to update the language since there are some archaic words.

Keeping the meter is important; rhyming is optional.

The distinction between the second person singular and second person plural is maintained as it is a characteristic of the Hebrew. The table below summarizes the forms.

\begin{longtable}[]{@{}ccc@{}}
\toprule\noalign{}
Function & Singular & Plural \\
\midrule\noalign{}
\endhead
\bottomrule\noalign{}
\endlastfoot
Subject & thou & ye \\
Possessive & thy, thine & yours \\
Object & thee & you \\
\end{longtable}

Note that the rules for \emph{thy} and \emph{thine} are the same as for the indefinite article \emph{a} and \emph{an}.  

This distinction between the singular and the plural may take some getting used to, but it seems to me to be a healthy matter for the Christian to continue to learn; particularly those matters that are present in the original languages of the Bible.  It also makes explicit the who is being spoken to and may suggest useful distinctions between the individual worshipper and the corporate aspects of worship.

